\documentclass[11pt,letterpaper]{article}

\usepackage[utf8]{inputenc}
\usepackage[spanish,es-nodecimaldot]{babel}
\usepackage{fullpage} % Package to use full page
\usepackage{parskip} % Package to tweak paragraph skipping
\usepackage{tikz} % Package for drawing
\usepackage{amsmath}
\usepackage{amsfonts}
\usepackage{amssymb}
\usepackage{hyperref}

\title{Práctica 1}
\author{Candela Castellanos Pablo Antonio\\
\and Hurtado Gutiérrez Marco Antonio\\ 
\and Hernández Austria Hugo Alexis\\}
\date{18 de febrero}

\begin{document}

\maketitle

\section{Pruebas}

\begin{enumerate}
\item
\begin{itemize}
\item
Hurtado Gutiérrez Marco Antonio
\item
Laptop Dell Inspiron 14-3451
\item
Procesador: Intel Pentium CPU N3540  @ 2.16GHz x 4 
\item
Memoria RAM: 3.7GB
\item
Disco duro: 476.6 GB
\item
Sistema operativo: Linux Mint 1 Cinnamon 64-b
\item
Versión del Kernel:4.40-62-generic
\end{itemize}

\item
\begin{itemize}
\item
Hernández Austria Hugo Alexis
\item
Laptop Lenovo Yoga 300
\item
Procesador: Intel Pentium N3700 @ 2.40GHz (4 núcleos) 
\item
Memoria RAM: 4096 MB Memoria cache: 2 MB
\item
Disco duro: 500 GB Seagate 5400 rpm
\item
Sistema operativo: elementary 0.4
\item
Versión del Kernel: 4.40-62-generic
\end{itemize}



\item
\begin{itemize}
\item
Candela Castellanos Pablo Antonio
\item
Laptop Dell Inspiron 1011
\item
Intel Atom CPU N270 @ 1.60GHz (2 nucleos)  
\item
Memoria RAM: 1 GB
\item
Disco duro: 476.6 Gb
\item
Sistema operativo: 32 bits
\item
Distribucion linux : Ubuntu 14.04
\item
Versión del Kernel:4.40-62-generic
\end{itemize}
\end{enumerate}

\section{Exploring the derivative using Sage}

The definition of the limit of $f(x)$ at $x=a$ denoted as $f'(a)$ is:

\begin{equation}
f'(a) = \lim_{h\to0}\frac{f(a+h)-f(a)}{h}
\end{equation}

The following code can be used in sage to give the above limit:

\begin{verbatim}
def illustrate(f, a):
    """
    Function to take a function and illustrate the limiting definition of a derivative at a given point.
    """
    lst = []
    for h in srange(.01, 3, .01):
        lst.append([h,(f(a+h)-f(a))/h])
    return list_plot(lst, axes_labels=['$x$','$\\frac{f(%.02f+h)-f(%.02f)}{h}$' % (a,a)])
\end{verbatim}

\begin{figure}[!htbp]
\begin{center}
\includegraphics[width=8cm]{sage1.png}
\end{center}
\caption{The derivative of $f(x)=1-x^2$ at $x=.5$ converging to -1 as $h\to0$.}
\end{figure}

If we want to plot the tangent at a point $\alpha$ to a function we can use the following:

\begin{align}
y=&ax+b&&\text{(definition of a straight line)}\nonumber\\
  &f'(a)x+b&&\text{(definition of the derivative)}\nonumber\\
  &f'(a)x+f(a)-f'(a)a&&\text{(we know that the line intersects $f$ at $(a,f(a))$}\nonumber
\end{align}

We can combine this with the approach of the previous piece of code to see how the tangential line converges as the limiting definition of the derivative converges:

\begin{verbatim}
def convergetangentialline(f, a, x1, x2, nbrofplots=50, epsilon=.1):
    """
    Function to make a tangential line converge
    """
    clrs = rainbow(nbrofplots)
    k = 0
    h = epsilon
    p = plot(f, x, x1, x2)
    while k < nbrofplots:
        tangent(x) = fdash(f, a, h) * x + f(a) - fdash(f, a, h) * a
        p += plot(tangent(x), x, x1, x2, color=clrs[k])
        h += epsilon
        k += 1
    return p
\end{verbatim}

The plot shown in Figure \ref{lines} shows how the lines shown converge to the actual tangent to $1-x^2$ as $x=2$ (the red line is the `closest' curve).

\begin{figure}[!htbp]
\begin{center}
\includegraphics[width=8cm]{sage0.png}
\end{center}
\caption{Lines converging to the tangent curve as $h\to0$.}\label{lines}
\end{figure}

Note here that the last plot is given using the \textbf{real} definition of the derivative and not the approximation.

\section{Conclusions}

In this report I have explored the limiting definition of the limit showing how as $h\to 0$ we can visualise the derivative of a function. The code involved \url{https://sage.maths.cf.ac.uk/home/pub/18/} uses the differentiation capabilities of Sage but also the plotting abilities.

There are various other aspects that could be explored such as symbolic differentiation rules. For example:

$$\frac{dx^n}{dx}=(n+1)x^{n}\text{ if }x\ne-1$$

Furthermore it is interesting to not that there exists some functions that \textbf{are not} differentiable at a point such as the function $f(x)=\sin(1/x)$ which is not differentiable at $x=0$. A plot of this function is shown in Figure \ref{notdiff}.

\begin{figure}[!htbp]
\begin{center}
\includegraphics[width=8cm]{sage2.png}
\end{center}
\caption{None differentiable function at $x=0$.}\label{notdiff}
\end{figure}


\bibliographystyle{plain}
\bibliography{bibliography.bib}
\end{document}